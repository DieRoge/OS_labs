\section{Метод решения}
Данная программа реализует многопоточную обработку матрицы вещественных чисел с использованием стандартных средств многопоточности языка C++11. Главный процесс создаёт заданное пользователем количество рабочих потоков, которые параллельно выполняют операцию свёртки для различных строк матрицы. Для синхронизации потоков и исключения гонок данных используется атомарная переменная, гарантирующая безопасное распределение строк между потоками.

{\bfseries Основные компоненты:}

\texttt{main.cpp} — главный файл, выполняющий ввод параметров, инициализацию матриц, запуск многопоточной обработки и вывод результата;
\texttt{matrix.h / matrix.cpp} — реализация класса матрицы, включающего хранение данных, доступ к элементам и вычисление свёртки для одного элемента;
\texttt{threads.h / threads.cpp} — реализация класса ThreadControl, отвечающего за создание потоков, распределение строк и организацию K-кратного применения фильтра;
\texttt{CMakeLists.txt} — файл конфигурации системы сборки CMake.

\section{Описание программы}

{\bfseries Структура проекта:}

lab1/

\hspace{3em}report/
    
\hspace{6em}...
        
\hspace{3em}include/
    
\hspace{6em}matrix.h    // Заголовочный файл класса Matrix

\hspace{6em}threads.h    // Заголовочный файл класса потоков
         
\hspace{3em}src/
    
\hspace{6em}matrix.cpp  // Реализация класса Matrix
        
\hspace{6em}threads.cpp         // Реализация управления потоками
        
\hspace{3em}CMakeLists.txt

\hspace{3em}main.cpp    

{\bfseries Основные типы данных:}

\textbf{1. Класс \texttt{Matrix}} \\
Класс предназначен для хранения и обработки двумерной матрицы вещественных чисел. 
Основные характеристики:
\begin{itemize}
    \item данные хранятся в контейнере \texttt{std::vector<std::vector<double>>};
    \item в объекте сохраняются размеры матрицы (число строк и столбцов);
    \item предоставляются методы доступа к элементам;
    \item реализована функция \texttt{applyConvolutionAt()}, вычисляющая значение фильтра свёртки в заданной точке;
    \item реализованы методы ввода и вывода матрицы.
\end{itemize}

\textbf{2. Атомарный счётчик строк (\texttt{std::atomic<int>})} \\
В программе используется атомарная переменная вместо семафоров и мьютексов. 
Она обеспечивает корректное распределение строк матрицы между потоками. 

Операция:
\begin{verbatim}
row = nextRow.fetch_add(1);
\end{verbatim}

выполняется атомарно и гарантирует, что каждая строка будет обработана строго одним потоком.

\textbf{3. Класс \texttt{ThreadControl}} \\
Класс отвечает за организацию многопоточной обработки матрицы. Основные обязанности:
\begin{itemize}
    \item создание и запуск рабочих потоков (\texttt{std::thread});
    \item распределение строк между потоками с помощью атомарного счётчика;
    \item выполнение K итераций свёртки;
    \item ожидание завершения потоков через \texttt{join()};
    \item обмен буферов матриц между итерациями с помощью \texttt{std::swap()}.
\end{itemize}

\textbf{4. Вектор потоков \texttt{std::vector<std::thread>}} \\
Используется для хранения и управления всеми рабочими потоками.  
После завершения работы потоки корректно ожидаются методом \texttt{join()}.

\section*{Принцип работы с типами данных}

В начале работы создаются две матрицы-буфера: \texttt{bufferA} (входные данные) и \texttt{bufferB} (результат свёртки). Далее выполняются несколько итераций:

\begin{enumerate}
    \item Инициализируется атомарный счётчик строк \texttt{nextRow = 0}.
    \item Создаётся заданное пользователем число потоков.
    \item Каждый поток получает строку для обработки с помощью \texttt{fetch\_add()}.
    \item Поток вычисляет свёртку для всех элементов строки и записывает результат в выходную матрицу.
    \item После обработки всех строк потоки завершаются.
    \item Буферы матриц меняются местами, и начинается следующая итерация.
\end{enumerate}

Использование атомарной переменной делает работу потоков безопасной и устраняет необходимость в семафорах или других средствах блокировки.

\section*{Основные функции программы}

\begin{itemize}
    \item \texttt{Matrix::applyConvolutionAt()} — вычисляет значение фильтра свёртки для элемента матрицы;
    \item \texttt{ThreadControl::applyConvolution()} — организует многопоточную обработку матрицы и выполняет K итераций;
    \item \texttt{ThreadControl::workerFunc()} — функция, выполняемая каждым рабочим потоком;
    \item \texttt{std::thread::join()} — ожидание завершения потока;
    \item \texttt{std::atomic::fetch\_add()} — атомарное распределение строк между потоками.
\end{itemize}

\section*{Алгоритм распределения работы}

Распределение строк матрицы между потоками выполняется динамически при помощи атомарного счётчика:
\begin{enumerate}
    \item Каждая итерация обработки начинается с установки \texttt{nextRow = 0}.
    \item Поток получает строку:
\begin{verbatim}
row = nextRow.fetch_add(1);
\end{verbatim}
    \item Если \texttt{row < totalRows}, поток обрабатывает эту строку.
    \item Если строк больше нет, поток завершает работу.
    \item После завершения всех потоков главный процесс выполняет обмен буферов.
\end{enumerate}

Такой алгоритм предотвращает гонки данных и обеспечивает равномерное распределение нагрузки между потоками.