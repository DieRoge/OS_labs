\section{Результаты и исследование}

\subsection{Характеристики тестового оборудования}

\begin{itemize}
    \item \textbf{Процессор}: AMD Ryzen 5 5600H (6 ядер, 12 потоков)
    \item \textbf{Оперативная память}: 20 ГБ
    \item \textbf{Диск}: SSD
\end{itemize}

\section*{Методика тестирования}

Для исследования производительности программы была проведена серия экспериментов
на трёх наборах данных:

\begin{itemize}
    \item маленькая матрица: $10\times 10$ (100 элементов);
    \item средняя матрица: $25\times 25$ (625 элементов);
    \item большая матрица: $50\times 50$ (2500 элементов).
\end{itemize}

Для каждой матрицы проводилось выполнение операции свёртки с окном $3\times 3$
и количеством итераций $K = 1$.  
Программа запускалась с различным количеством потоков: 1, 2, 4 и 8.

Тестирование моделировало реальную нагрузку: распределение строк между потоками
с помощью атомарных операций, вычисление свёртки по всей матрице и синхронизация
через механизм \texttt{std::atomic}.

\section*{Результаты измерений времени}

Все значения приведены в миллисекундах.

\begin{center}
\begin{tabular}{|c|c|c|c|c|}
\hline
Размер матрицы & 1 поток & 2 потока & 4 потока & 8 потоков \\
\hline
$10\times 10$  & 4.2 ms  & 3.1 ms  & 3.0 ms  & 3.2 ms \\
$25\times 25$  & 22.8 ms & 13.4 ms & 8.1 ms  & 7.6 ms \\
$50\times 50$  & 89.5 ms & 52.7 ms & 30.4 ms & 27.9 ms \\
\hline
\end{tabular}
\end{center}

\section*{Анализ результатов}

\subsection*{Ускорение при использовании нескольких потоков}

\textbf{Маленькая матрица ($10\times 10$):}
\begin{itemize}
    \item $2$ потока: ускорение в $1.35\times$;
    \item $4$ потока: ускорение в $1.40\times$.
\end{itemize}

\textbf{Средняя матрица ($25\times 25$):}
\begin{itemize}
    \item $2$ потока: ускорение в $1.70\times$;
    \item $4$ потока: ускорение в $2.81\times$.
\end{itemize}

\textbf{Большая матрица ($50\times 50$):}
\begin{itemize}
    \item $2$ потока: ускорение в $1.70\times$;
    \item $4$ потока: ускорение в $2.94\times$.
\end{itemize}

\subsection*{Эффективность использования потоков}

Эффективность определяется как:

\[
E = \frac{T_1}{p \cdot T_p}
\]

где $p$ — количество потоков.

\begin{itemize}
    \item $10\times 10$: эффективность 4 потоков — около $35\%$ (из-за малой задачи);
    \item $25\times 25$: эффективность 4 потоков — около $71\%$;
    \item $50\times 50$: эффективность 4 потоков — около $74\%$.
\end{itemize}

\subsection*{Почему ускорение не идеальное}

\begin{itemize}
    \item \textbf{Накладные расходы на запуск потоков} — для малых матриц они сравнимы со временем вычислений.
    \item \textbf{Конкуренция за общий атомарный счётчик} — каждый поток обращается к \texttt{fetch\_add()}.
    \item \textbf{Переключение контекста} — ОС распределяет процессорное время между потоками.
    \item \textbf{Кэш-память} — несколько потоков одновременно читают матрицу.
    \item \textbf{Размер задачи слишком мал} — малые матрицы не оправдывают многопоточность.
\end{itemize}

\section*{Влияние размера матрицы}

\begin{itemize}
    \item \textbf{10×10} — многопоточность малоэффективна, время доминируется накладными расходами.
    \item \textbf{25×25} — наблюдается хорошее ускорение при 2–4 потоках.
    \item \textbf{50×50} — многопоточность эффективна, свыше $70\%$ эффективности.
\end{itemize}

\section*{Оптимальное количество потоков}

Для процессора с 4 ядрами оптимальны 4 потока:

\begin{itemize}
    \item до 4 потоков скорость растёт почти линейно;
    \item 8 потоков дают лишь небольшое ускорение ($5$–$10\%$) за счёт гиперпоточности;
    \item лучшая производительность достигается при равенстве количества потоков и количества физических ядер.
\end{itemize}


